\documentclass{resume}
\usepackage[hidelinks]{hyperref}
\usepackage{lipsum}

\name{Jiale GUO}

\contact{\href{jiale.guo@mail.polimi.it}{jiale.guo@mail.polimi.it}}

\contact{\href{https://github.com/GuojialeGeographer}{GuojialeGeographer/Github}}


\begin{document}

\makeheader

\begin{ResumeSection}{EDUCATION EXPERIENCE}
    \begin{ResumeSubsection}{org={Politecnico di Milano (Italy), M.S. in Geoinformatics Engineering}, location={\textit{09.2024 - Present}}}
    
    \begin{ResumeSubsection}{org={ChuZhou University (China), B.A. in Geography}, location={\textit{09.2017 - 06.2021}}}



%\textbf{GPA:} 84.85/100; \textbf{GPA:} 84.85/100; \textbf{Ranking:} 3/60 (2017 - 2018) ; 8/56 (2018 - 2019) ; 1/56 (2019 - 2020)

    \end{ResumeSubsection}
\end{ResumeSection}

\vspace{-6mm}  
\begin{ResumeSection}{RESEARCH EXPERIENCE}

\textit{\textbf{Runoff simulation and prediction in the Qinghai Lake basin based on the SWAT model} }

Research Assistant (Chinese Academy of Sciences) \hfill{\textit{10.2021 - 06.2022}} 

    \begin{itemize}
        \item Simulated and predicted multi-year variation rules of water resources in the Qinghai Lake Basin based on the SWAT model.
    \end{itemize}

    \begin{itemize}
    
        \item Performed data analysis and mapping by leveraging remote sensing for data analysis and interpretation.
    \end{itemize}


 

\textit{\textbf{Spatial-temporal Evolution and Trend Analysis of Drought Disasters in the Jianghuai Watershed Region} }

Undergraduate Thesis (Chuzhou University)	 \hfill{\textit{01.2021 - 05.2021}} 

    \begin{itemize}
    
        \item Computed the Standardized Precipitation Evapotranspiration Index (SPEI) as an indicator for drought evaluation.
    \end{itemize}

    \begin{itemize}
    
        \item Explored the spatial-temporal change characteristics and future trends of meteorological drought in the Jianghuai watershed area based on EOF, M-K test, Wavelet Analysis, and Spatial Analysis methods.
    \end{itemize}



\textit{\textbf{Spatial-temporal pattern mining of PM2.5 pollution in China and analysis of its driving mechanisms} }

Research Assistant (Geoinformation Lab, Chuzhou University)	 \hfill{\textit{01.2020 - 06.2021}} 

    \begin{itemize}
    
        \item Leveraged 3D GIS technology and the space-time cube model, and employed spatial-temporal hotspot data analysis methods, to examine the spatial-temporal evolution trends and distribution patterns of PM2.5 in China. 
    \end{itemize}

    \begin{itemize}
    
        \item The impacts of natural and socio-economic factors on PM2.5 using the Geodetector and Geographically Weighted Regression models.
    \end{itemize}

\textit{\textbf{On the Remote Sensing Classification of Tree Species Based on Multi-scale Feature Transfer Learning} }

Research Assistant (Remote Sensing and Big Data Analysis Lab, Chuzhou University)	 \hfill{\textit{06.2018 - 12.2019}} 

    \begin{itemize}
    
        \item Utilized QuickBird high-resolution remote sensing imagery, a tree species classification model with high accuracy was developed.

    \end{itemize}

    \begin{itemize}
    
        \item Combined multi-scale segmentation algorithms and transfer learning to achieve high-precision identification and classification of forest tree species.

    \end{itemize}



\vspace{1mm}  
\begin{ResumeSection}{WORK EXPERIENCE}

\textbf{\textit{Industrial Park Surveying and Mapping Geographic Information Technology Co., Ltd. }}\hfill{Suzhou, China}

{Geographic Information Data Engineer }
\hfill{\textit{01.2023 - Present}} 
    \begin{itemize}
    
        \item Manipulate high-definition remote sensing images of suzhou City, to determine the scope of forests, grasslands, and wetlands, and carry out the segmentation and processing of the identified patches.

    \end{itemize}

    \begin{itemize}
    
        \item Inductive reasoning, analysis, and understanding of natural resource data, such as land use data and remote sensing images, conduct to design innovative solutions based on practical needs.

    \end{itemize}




\textbf{\textit{Chuzhou Municipal Public Utility Engineering Co., Ltd. (Urban Water Supply)}}\hfill{{Chuzhou, China}}

{Graduation Internship}
\hfill{  \textit{03.2021 - 05.2021}} 
    \begin{itemize}
    
        \item Used GPS to collect geospatial positional data of urban drainage pipes, perform data cleaning and preprocessing, conduct geospatial buffer analysis, and create map visualizations.

    \end{itemize}

    \begin{itemize}
    
        \item Analyzed the residential coverage of the current plan of pipelines, and designed a reasonable geospatial optimum coverage scheme.

    \end{itemize}


\textbf{\textit{Anhui Engineering Laboratory of Geo-information Smart Sensing and Services}}\hfill{{Chuzhou, China}}

{Internship}
\hfill{\textit{05.2019 - 12.2019}} 
    \begin{itemize}
    
        \item Utilized ArcGIS, CAD, for spatial data collection, processing, analysis, mapping.

    \end{itemize}

    \begin{itemize}
    
        \item Conducted aerial image data collection and data processing using unmanned aerial vehicles (drones).

    \end{itemize}



\vspace{1mm}  
\begin{ResumeSection}{PEER-REVIEWED PUBLICATIONS}
• Li, J., Li, R., Zhang, M., \textbf{\textit{Guo, J.,}} Shi, L., \& Guo, A. (2021). Study on characteristics of temporal and spatial evolution of rural settlements in typical land consolidation counties--Taking Dingyuan County of Anhui Province as an example. \textit{Journal of Jiangsu Agricultural Sciences} (17), 202-208. In Chinese \textit{Doi:10.15889/j.issn.1002-1302.2021.17.036.}

• Wang, N., Min, J., \textbf{\textit{Guo, J.,}} \& He, N. (2021). On the Remote Sensing Classification of Tree Species Based on Multi-scale Feature Transfer Learning. \textit{Journal of Ezhou University} (02), 93-97. In Chinese \textit{Doi:10.16732/j.cnki.jeu.2021.02.031.}

• Cao, X., Zhou, L., Dai, S., \textbf{\textit{Guo, J.,}} \& Ju, X. (2020). Analysis of spatial pattern and effect of the farm pond system in Jianghuai water-shed area. \textit{Journal of Heilongjiang Institute Of Technology} (06), 14-22. In Chinese \textit{Doi:10.19352/j.cnki.issn1671-4679.2020.06.003.}


\end{ResumeSection}



\vspace{-2mm}  
\begin{ResumeSection}{HONORS \& COMPETITIONS}

• "Esri" College Students GIS Software Development Competition in China, Third Prize \hfill{2022}

• GIS Skills Competition for College Students in Anhui Province, Second Prize \hfill{2019 - 2020}

• The 5th China University Geography Science Exhibition Competition, Third Prize \hfill{2019}

• Excellent Graduation Thesis of Chuzhou University (Top 3\%) \hfill{2021}

• Excellent Graduates of Ordinary Colleges and Universities in Anhui Province (Top 2\%) \hfill{2021}

• National Encouragement Scholarship, Education Department of Anhui Province (Top 3\%) \hfill{2020}

• Excellent Interns \& Internship Works (Top 10\%) \hfill{2019}

• Academic Excellence Scholarship (Three consecutive years) \hfill{2017 - 2020}

• Outstanding Student Cadres (Three consecutive years) \hfill{2017 - 2020}

\end{ResumeSection}






    
\end{ResumeSection}

\vspace{-6mm}  
\begin{ResumeSection}{Other SKILLS}

\textbf{Programming:} Python, R, LaTex

\textbf{Software:} ArcGIS Pro, Google Earth Engine, QGIS, PostgreSQL	

\textbf{Languages:} English (fluent)
\end{ResumeSection}



\end{document}
